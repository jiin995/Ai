\raggedright
\mychapter{2}{Work Project 2}
	\label{ch:opt}
	\section{Esercizio 1: Evolutionary Optimization}
		\label{sec:es1}
		\subsection{Esperimento 1: Ottimizzazione}
			Attraverso gli algoritmi forniti siamo stati in grado di sperimentare le differenti tecniche di \emph{ricerca locale} su diversi problemi. Esse si basano sul principio dei miglioramenti successivi: si cerca in primo luogo una soluzione ed in seguito ci si concentra sull'ottimizzazione. Va considerato che spesso si riesce a determinare solo un massimo locale, ottenendo risultati più o meno soddisfacenti a seconda dell'applicazione.
			\subsubsection{Random Searching}
				Questo algoritmo prevede semplicemente la determinazione casuale di un insieme di soluzioni e la comparazione dei relativi costi per l'identificazione di quella a costo minimo.\par
				Abbiamo effettuato differenti esperimenti variando il numero di soluzioni estratte casualmente: ad un suo eccessivo aumento non si ottengono necessariamente risultati nettamente migliori da giustificare l'evidente incremento della complessità computazionale in termini temporali; d'altro canto, un numero esiguo di soluzioni produce risultati dalla qualità incostante.
			\subsubsection{Hill Climbing}
				L'\texttt{Hill Climbing} inizialmente determina una soluzione ed i suoi "vicini": se uno di essi presenta un costo minore della soluzione corrente, diventerà la nuova soluzione alla prossima iterazione, attuando, in tal modo, il processo dei miglioramenti successivi; se la soluzione corrente ha costo minore dei suoi vicini, termina la serie di miglioramenti iterativi poiché si è in presenza di un massimo locale (minimo locale rispetto al costo).\par
				Vari esperimenti ci hanno condotto a risultati decisamente migliori del \texttt{Random Searching}, che sottolineano la maggior efficienza dell'\texttt{Hill Climbing}. La variabilità nella qualità della soluzione, al seguito di svariate esecuzioni, è dipendente dai massimi locali del problema specifico.
		\subsection{Esperimento 2: Caso d'uso}
	\section{Esercizio 2}
		\label{sec:es2}
	\section{Esercizio 3}
		\label{sec:es3}
	\section{Esercizio 4}
		\label{sec:es4}