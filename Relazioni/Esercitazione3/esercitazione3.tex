\raggedright
\mychapter{3}{Work Project 3}
	\label{ch:opt}
	\section{Esercizio 1: Logica Proposizionale}
		\label{sec:es1}
		Un agente intelligente che sfrutta tale approccio, si rifà al comportamento umano di ragionare su sue conoscenze personali. Quindi ha bisogno di una base di conoscenza in cui archiviare tutte queste informazioni, affinché possa usarle per trarne delle deduzioni.
		\par
		L'agente semplicemente ogni volta che ha bisogno di dedurre qualcosa, prima di tutto enuncia alla sua base di conoscenza qual'è la sua percezione in questo momento, dopodiché chiede a quest' ultima quale operazioni deve eseguire in base alla informazioni archiviate all' interno e comunica  di aver eseguito l' operazione. Ovviamente l' agente all' inizio non è a conoscenza di tutti i fatti che gli occorrono per completare la sua operazione, per tale ragione può raccogliere informazioni chiedendo i fatti(approccio dichiarativo) di cui gli interessa o in altri casi può avere già della conoscenza archiviata (approccio procedurale), di solito i due approcci vanno usati insieme per avere una buona base di conoscenza.
		\par
		Ovviamente tali informazioni non possono essere salvate nella conoscenza dell' agente in un linguaggio naturale o comunque in un qualsiasi modo in cui non sia possibile trarre delle conclusioni. Per tale motivo le frasi che definiscono la nostra conoscenza sono scritte con una certa sintassi, propria del linguaggio che vogliamo utilizzare, la stessa identica frase espressa naturalmente può essere tradotto in modi differenti in base alla diversa sintassi della logica che andiamo ad utilizzare. Però il risultato da ottenere sarà lo stesso che è indipendente dalla logica utilizzata, cioè avendo dei dati noti a priori, capire se si può riuscire a dedurre altre informazioni che a noi interessa sapere, tale relazione è detta di implicazione(entailment). Utilizzeremo la logica proposizionale, in cui la sintassi è composta da frasi atomiche, legate tra loro tramite connettivi logici per ottenere frasi più complesse, il singolo elemento della frase può assumere valore o vero o falso,infatti la semantica di tale logica è quella di trovare il valore vero di ogni frase dato la conoscenza nota a priori. Le operazioni tra i letterali sono: la negazione che muta il valore della nostro letterale, l' operazione di and che ci da vero solo se i due connettivi sono veri, or restituisce vero se uno dei due è vero, implicazione ci dà falso se dal vero deduciamo il falso e quella bicondizionale vera se e solo se entrambi i letterali hanno lo stesso significato.
		Per verificare la relazione di implicazione possiamo ricorrere a due strade la prima in cui enunciamo tutti i possibili mondi del nostro modello, per esempio tramite la tabella di verità,trovare quelli che rispettano la nostra base di conoscenza e la deduzione che stiamo facendo e se i mondi che rispecchiano la nostra percezione sono contenuti all' interno della concetto che vogliamo dedurre, allora possiamo dire che la nostra conoscenza implica quel fatto, sulla tabella li possiamo vedere facilmente osservando che quando sono veri i fatti allora anche ciò che vogliamo dedurre è vero.
		\par
		Questo metodo è molto oneroso quando i possibili mondi sono tanti ed enumerarli tutti sarebbe molto dispendioso sia in termini di spazio che di tempo, per tale ragione di preferiscono altre strade.
		\par
		Una possibile è scrivere le frasi in forma CNF e da li applicare metodi inferenziali per capire se la nostra conoscenza implica il fatto. 
		Prima di tutto però dobbiamo trasformare le frasi che abbiamo nella forma CNF, cioè or di congiunzioni atomiche, per farlo sfruttiamo le regole di equivalenza logica, purtroppo anche qui c'è un grande sforzo dell' essere umano che deve applicare tali regole affinché sia possibile dedurre qualcosa.
		Quando la nostra conoscenza è nella forma da noi voluta, possiamo provare a dimostrare che la nostra conoscenza con il negato della frase che vogliamo dedurre non sia soddisfacibile, cioè non esista nessun modello che implica quella frase, se ciò accade significa che la frase non negata è valida, cioè in tutti i mondi in cui è vera la nostra base di conoscenza è vera anche la frase, per verificare che not A sia soddisfacibile, basta prendere una coppia di clausole(congiunzione di letterali), presenti nella nostra base di conoscenza ed iniziare a derivare nuove informazioni,ciò avviene quando nella frase che si vengono a creare si presentino due letterali complementari (A e not A),
		applichiamo questo ragionamento ricorsivamente fino a che non riusciamo più a determinare nuove clausole e quindi quel fatto non è deducibile dalla nostra conoscenza, oppure se si arriva alla clausola nulla che ci permette di dire che la not A non è soddisfacibile allora A è valida.
		\par
		Tutti e due algoritmi sono completi e soundness.
		Un algoritmo è completo se é capace di enumerare tutte le relazione di implicazione possibili dalla base di conoscenza, ed è soundness perché ogni implicazione dell' algoritmo sono anche vere nel caso della base di conoscenza.
		Per entrambe le procedure enunciate vale il discorso che l' agente non conosce il significato della frase, quello è dato dall' essere umano, per l' agente una frase vale un' altra è solo capace di applicare le regole inferenziali